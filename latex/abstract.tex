Since the advent of modern digital communications in the 20th century, there has been an explosion in the demand for wireless spectrum. As a result, spectrum is becoming an increasingly scare resource. This demand is a direct result of the availability and relatively inexpensive cost of such wireless devices. Therefore, in such environments as military operations, disaster relief scenarios, and natural defense situations, the probability of interfering transmissions \cite{scarcity}, intended and unintended, has steadily grown to a point where techniques are needed in-order to combat such occurrences. More directly, in such situations when interfering signals are partially or completely understood measures need to be devised in-order to overcome such difficulties. Under these assumptions, several well know techniques are can be applied to combat such scenarios.  This research analyzes the feasibility of combining Antenna Subset Selection, Spectral Subtraction, and Blind Source Separation signal processing techniques to accomplish this goal.  Together they provide multiple avenues of signal separation to remove such jamming effects, for both narrow and wide bands, without hindering the mobility of the nodes themselves.
