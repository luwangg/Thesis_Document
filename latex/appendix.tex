\chapter{Appendix B: SS_correlate.m}

%Spectral Subtraction Implementation Using Pulse Shapes in Frequency Domain
clear all; close all; clc;
% Half the length of the srrc pulse
size = 10;

% amount of information to be sent
dataLength = 4e3;

% the signal to be transmitted
%data = 2*randint(dataLength,1)-1;
data = 2*(randn(dataLength,1)>0)-1;


data2 = randi([-3,3], dataLength/4,1);
data2=[ data2; data2; data2; data2];

% upsamp_data is the signal after it has ben upsampled by 10
upsamp_data = upsample(data,10);
upsamp_data2 = upsample(data2,10);
% square root raised cosine pulse in the time domain
samples = 10;   % Number of samples
Beta_rolloff=.5;    % roll off factor for the srrc pulse

% pulse_srrc is the srrc pulse
pulse_srrc = 10*srrc(size,Beta_rolloff,samples);

% x_data is the signal convolved with the pulse shape
x_data = conv(pulse_srrc,upsamp_data);

% Carrier frequency and modulation information
fo1 = 8e4;
Fs = 2e5;
t = 1/Fs:1/Fs:length(x_data)/Fs;
x_Modul = x_data.*cos(2*pi*t*fo1)';

% pulse_hamming is a hamming pulse in the time domain
pulse_hamming = hamming(size*samples*2+1);

% y_data is the interfering signal
y_data = conv(pulse_srrc,upsamp_data2);

index=0;
BitError=zeros(1,401);
for int_freq=-20:0.1:20
% Carrier frequency for interfering signal
fo2 = 8e4-1e3*int_freq;

y_Modul = 2*y_data.*cos(2*pi*t*fo2)';

% z is the combination of x_data and y_data providing the interference
z = x_Modul+y_Modul;

% Here the noise is added to the signal
m = awgn(z,10,'measured');


% this defines the precision of the fft
precision = 20*dataLength;

%% Correlate and remove
%generate subtraction length
average_power=mean(abs(y_Modul));
seq_len=length(m);
extra=mod(seq_len,length(y_Modul));
segments=floor(seq_len/length(y_Modul));
SS=[];
for i=1:segments SS=[SS y_Modul]; end;
if extra>0
SS=[SS y_Modul(1:extra)];
end
x_SS = m - SS;


%% The signal is modulated back down to base band
t2 = 1/Fs:1/Fs:length(x_SS)/Fs;
X_BB = 2*x_SS.*cos(2*pi*t2*fo1)';

% A filter is applied to remove any unwanted information outside the
% desired signal

fl=600; 
ff=[0 .1 .11 1];
fa=[1 1 0 0];
h=firpm(fl,ff,fa);
X_filt = filter(h,1,X_BB);

% x_SS_data is retrieved signal
x_SS_data = 2*downsample(conv(X_filt, pulse_srrc),10)/100;

numErrors = 0;

for i=51:length(data)
    if x_SS_data(i)>0
        x_SS_data(i) = 1;
    else
        x_SS_data(i) = -1;
    end
    if x_SS_data(i)~=data(i-50)
        numErrors=numErrors+1;
    end
end

index=index+1;
BitError(index) = numErrors/length(data);
disp(numErrors/length(data));

end
% Plot
int_freq=-20:0.1:20;
plot(int_freq*10e3,BitError);
ylabel('BitError');
xlabel('Interferer Frequency Offset');
grid on;
title('SS Correlation with Varying Interfering Offset Frequency')
break

% Here the process is plotted
figure(1),
subplot(5,1,1),
plot(upsamp_data),
title('Original Data');
subplot(5,1,2),
plot(abs(fft(x_Modul))),
title('Desired Signal');
subplot(5,1,3),
plot(abs(M_fft)),
hold on
plot(abs(fft(y_Modul)),'r')
hold off;
title('Combined Signals');
subplot(5,1,4),
plot(abs(X_fft)),
title('Subtracted result');
subplot(5,1,5),
plot(real(x_SS_data(51:dataLength+50))),
title('Time Domain Data w/ Pulse Shape');

% This figure shows the retrieved signal in blue and the orriginal in red
figure(2)
plot(data,'r.')
hold on
plot(real(x_SS_data(51:dataLength+50)),'.')
title('Comparison of Original and Retrieved Data');

