%%%%%%%%%%%%%%%%%%%%%%%%%%% START WITH LATEX PREAMBLE %%%%%%%%%%%%%%%%%%%%%%
% reminder: its = possessive
%           it's = it + is

%use [11pt, draft] for draft markings.
\documentclass[11pt]{mvlthesis}
\usepackage[dvipdfm]{graphicx}
\usepackage{verbatim} %,amssymb,amsmath}
\usepackage{epstopdf}
%\usepackage{afterpage}


%%%%%%%%%%%%%%%%%%%%% SET UP ALL THE TITLE PAGE VARIABLES %%%%%%%%%%%%%%%%%%

\title{\scshape \mbox{THESIS OR DISSERTATION TITLE}\\
\scshape \mbox{TITLE LINE 2}}

\author{Travis F. Collins}
\thesis_or_diss{Thesis}
\degree_type{Master of Science}
\field{Electrical and Computer Engineering}
\degreeyear{December 2012}
\chair{Professor Alexander Wyglinski}
\chairtitle{Major Advisor}
\membertwo{Professor Y}
\memberthree{Professor Z}

%%%%%%%%%%%%%%%%%%%%%% INCLUDE USER DEFINED COMMANDS %%%%%%%%%%%%%%%%%%%%%%%

\newcommand{\bi}{\begin{itemize}}
\newcommand{\ei}{\end{itemize}}
\newcommand{\ii}{\item}
\newcommand{\be}{\begin{enumerate}}
\newcommand{\ee}{\end{enumerate}}
\newcommand{\ie}{\item}

\newcommand{\fig}[5]{
    \begin{figure}[#1]
    \begin{center}
    \includegraphics[#2]{#3}
    \end{center}
    \caption{#4}
    \label{#5}
    \end{figure}
}

%\frenchspacing

%%%%%%%%%%%%%% SPECIFY WHICH PARTS OF THE THESIS YOU WANT PRINTED %%%%%%%%%%

\renewcommand{\baselinestretch}{1.5}

\newcommand{\pderiv}[2]{\mbox{$\frac{\displaystyle \partial #1}{\displaystyle \partial #2}$}}

%%%%%%%%%%%%%%%%%%%% Done with setup, document starts here %%%%%%%%%%%%%%%%%
\begin{document}


%%%%%%%%%%%%%%%%%%%%%%%%%%%%%% TITLE + ABSTRACT %%%%%%%%%%%%%%%%%%%%%%%%%%%%
\maketitle
\begin{abstract}

Since the advent of modern digital communications in the 20th century there has been an explosion in the demand for wireless spectrum. As a result spectrum is becoming an increasingly scare resource. This demand is a direct result of the availability and relatively inexpensive cost of such wireless device. Therefore in such environments as militaristic theatres the probability of interfering transmissions, intended and unintended, has steadily grown to a point where techniques need to be consider to combat such occurrences. More directly, in such situations when interfering signals are partially or completely understood measures can be taken to overcome such difficulties. Under these assumptions several well know techniques are can be applied to combat such scenarios.  This research analyzes the feasibility of combining Antenna Subset Selection, Spectral Subtraction, and Blind Source Separation signal processing techniques to accomplish this goal.  Together they provide multiple avenues of signal separation to remove such jamming effects, for both narrow and wide bands, without hindering the mobility of the nodes themselves.


\end{abstract}

%%%%%%%%%%%%%%%%%%%%%% ACKNOWLEDGMENTS + TABLE OF CONTENTS %%%%%%%%%%%%%%%%%

\begin{frontmatter}

\begin{acknowledgements}
%\begin{center}
%\vspace{0.4in}
%\end{center}
\end{acknowledgements}
\tableofcontents
\listoffigures
\listoftables

\end{frontmatter}

%%%%%%%%%%%%%%%%%%%% INCLUDE THE REST OF THE DOCUMENT %%%%%%%%%%%%%%%%%%%%%%


\chapter{Introduction}
\label{ch:introduction}
\section{Motivation}

Since the advent of modern digital communications in the 20th century there has been an explosion in the demand for wireless spectrum.  As a result spectrum is becoming an increasingly scare resource(Insert citation).  This demand is a direct result of the availability and relatively inexpensive cost of such wireless device.  Therefore in such envirorments as militaristic theatres the probability of interfering transmissions has steadily grown to a point where techniques need to be consider to combat such occurrences.  More directly, in such situations when interfering signals are partially or completely understood measures can be taken to overcome such difficulties.\\

In military theatres it is extremely common to observe friendly operated high-power broadband jamming signals(citation).  Such devices exists as part of group convoys in several branches of the military and in many other forms in contested territories or war-zones.  Unfortunately such devices block both friendly and hostile communications, and current anti-jamming techniques haven't provided a viable solution to this problem.  Therefore new avenues should be considered, utilizing more flexible radio technologies.\\

Understanding how to overcome such challenges is a complex task; with vastly different transmission envirorments and differing operating devices and operating standards.  A new system that could combat such downfalls should rely on all friendly information, or be able to construct solutions of its own from a set of tools given to the radio.  Such tools should be flexible and easily modified, changed, or improved.  This ability to easily change or adapt is a key feature as the technical requirements can change from day to day, or between branches of the military itself. As such a solution should have the following attributes:

\begin{itemize}
\item \textbf{Flexible:}

\item \textbf{Resilient:}

\item \textbf{Hardened:} in changing enviroments

\end{itemize}


\section{State of the Art}

Current implementations in anti-jamming technology lies on the strateling point of hardware and software in the communications world.  This is true because hardware provides the speed and performance needed for digital data transmission, while software provides higher level intelligence and flexibility in such layers as the media access control layer and the network layer of the OSI model (insert citation of OSI model.)  For anti-jamming applications, high intelligence allows for mobility again the jammer.  Therefore a large implementation in software must be considered when investigating anti-jamming technics.\\

Insert figure of OSI model\\

Current anti-jamming technics include channel hopping, spatial retreat, jammed area mapping, node escape, retreat restoration, frame masking, and many more\cite{1}.  All of these techniques use mechanisms of evasion or despection.  These can be quite effective when attacked by generally narrowband, non-dynamic/learning jammers.  In the case of wide and ultra-wide band jammers, they fail miserably.  This wide-band enviorment is the primary situation of interest, and it generally considered a hopeless scenario.  These anti-jam technics are design for specific situations and jammers.\\

Let us first examine these anti-jamming technics which are broken down into three primary categories: Proactive countermeasures, Reactive countermeasures, and Mobile agent-base countermeasures\cite{1}.  Reactive countermeasures relies on a varying array of detection mechanisms first to determine if that node is being jammed.  These detection methods  must be coupled with a countermeasure or the scheme is in operable.  Examples of these detection methods include a transmitter-based approach and a receiver-based detection.  In a transmitter-based approach, such as ad-hoc networks, a decision algorithm is used based on four metrics: PDR (Packet Delivery Ratio), RSSI (Received Signal Strength Indicator), Physical rate, and Noise levels\cite{3}.  In the receiver-based detection additional information must be injected into frames to help the receiver determine the number of frames lost.  Since frames can be easily lost in wireless transmissions, the receiver is handicapped when determining the number of retransmissions that have occurred.  In the transmitter the PDR is deterministicly determined by the data-link layer, sequence numbers must be added to frames for the receiver to accurately calculate the PDR\cite{3}.  Several other detection methods exist including using a detected detector, cooperative detection among nodes in a wireless network, and more sophisticated methods of RF fingerprinting\cite{3}.\\

Once the jammer has been detected the reactive countermeasures come into play.  Many evasion techniques exists to combat narrowband jammers such as: channel hoping, spatial retreat, retreat restoration, hybrid attacks, and many cognitive radio approaches\cite{2}.  Many of these technics utilize the network itself to adapt to the jammer, which is an appropriate assumpt because without a network communications are irrelivant.  Channel hopping is quite simple and can be considered easiet to implement.  If a channel is begining jammed simply \"hop\" to another channel.  This is easily defeated in two cases, the first the jammer follows you or the jammer is simply wideband capable.  The second, spatial retreat, is a mechanism to physically evade the areas being jammed. Based on the detection algorithm all nodes in a network try to estimate the jammed region and flee physically in the direction of safer place. Based on their estimation about the jammed region, nodes will utilize shortest path algorithms to determine location of retreat\cite{5}.  Retreat restoration is focused around how to rebuild a network once the jammer has left.  Retreat restoration can be done by coordinated or uncoordinated communication, and the transmissions are based on a pre planned hop patterns among nodes\cite{6}.\\

There also exists systems that are design to resist jamming proactively.  These hybrid systems\cite{7} utilize preventatives measure to resist jamming such as frequency hopping spread spectrum \(FHSS\).  Spread-spectrum signals are highly resistant to narrowband jamming, unless the jammer has knowledge of the spreading key. In military applications the spreading key is generally created using a crytographic function(NEED citation).  More hybrid solutions include synchronous and asynchronous spectral multiplexing where intermediary nodes are used to communicate at multiple channels.  When a node changes its channel because of jamming a neighbor will heal that connection by communicating  with the node on its new channel and rest of the network on the old channel\cite{8}.


The largest problem with these techniques is they all have are designed to combat narrowband jammers, and even friendly jammers.  If high powered wideband jammers enter the equation, all of these solutions fall apart.  Note these techniques primarily exploit the dimensionality of their envirmonent by simply avoiding the jammer, and all techniques require intelligent flexible hardware solutions.   To implement such solutions requires suffisticated hardware implementations, that can be quite rigid for rapidly changing communication envirorments and adversaries.  To compensate solutions that push more of the radio operations from their original rigid hardware implementations into the more flexible software domain, provide a more cost effective and elligent solution.  These software focused radios, also know as Software defined radios, have provided a solid platform for very adaptive anti-jamming technologies under the name cogntive radios.  These radios have the ability to easily learn and adapt to their envirorment, which is the primary requirement of anti-jamming devices.\\

As mentioned above, it is quite common for the military to self-jam its own channels.  Unfortunately this can hinder their own use unintenially.  These disrupted users are known as "disadvantage users".  They are commonly small mobile hand held devices and cannot simply overcome the jammer computationally or in raw power; therefore, more manageable and eligent solutions must be considered for such disadvantaged users.  Beside self-jammming, adversarial jammers must also be considered.  Fortunately certain characteristics can be statistically exploited if these jammer abide by certain properties. Since adversarial jammers tend to inject random data or energy to block communication, if these transmissions can be shown to repeat they can be exploited.  In the case of self-jamming, the signal characteristic can be know \textit{a priori}; therefore they also can exploted or removed, negating the effects of such devices.  Such scheme must consider the energy or symbols of the jammmer that are orthoganol and/or non-orthoganol to the symbols of the communication itself.\\


The goal of this project is to exploit a self-jammed and statiscially determistic adversarially jammed channel, throught the utilization of cognitive radio, implemented on a software defined radio platform.  Software defined radios, defined as the intersection between hardware radios and computer software\cite{4}, provide a platform flexible enough to support highly intelligence operations such that anti-jamming requires.  A proposed adaptive signal processing software solution for mitigating the effects of both intentional and unintentional jamming (including wideband jamming) via the combination of antenna subset selection, spectral subtraction, and blind source separation (BSS) techniques in order to extract specific transmissions from a mixture of intercepted wireless signals. The goal of our proposed solution, called BLInd Spectrum Separation (BLISS), is to enable reliable, high throughput, and robust end-to-end wireless communications.\\

\section{Thesis Contributions}

This thesis will contribute the following to the wireless communications and signal processing research communities:

\begin{itemize}
\item A basis for blind source separation of define subset of signals, and tools on estimating and removing those signals.

\item A practical implementation using over the air communications of a anti-jamming sytem utilizing software defined radios. This implementation will tackle wideband non-orthoganol and orthoganol jamming, and provide evidence of probability of operational.
\end{itemize}


\section{Thesis Organization}

This thesis will be organized into the following chapters.  Chapter 2 provides the necessary background to understand basic communication system design, anti-jamming technics, and signal processing.  Chapter 3 puts forward a theoritical simulations and a design of a physical anti-jamming system.  Chapter 4 presents the results of the physical implement and analysis of its findings.  Chapter 5 concludes the thesis, summarizing the accomplishments and outlines possible future work.

\chapter{Background}
\label{ch:background}

This chapter provides the background information needed to understand the chapters that follow.  It examines the basic outline of a communication system and how non-idealities are compensated for, with addition of multiple input multiple output (MIMO) systems and a unique filtering technique called spectral subtraction.   Secondly this chapter investigates common jammer scenarios and anti-jamming solutions.  Finally it outlines the necessary hardware and software tools used during in the implementation chapter.

\section{Jamming}

In 1899 Guglielmo Marconi successful transmitted radio messages across the English Channel, and nine months later Alexander Bell was discussing how this could be jammed during wartime\cite{10}. Bell stated that such a wireless system can be easily disrupted with simple electromagnetic distrubances.  "Its as easy as cutting the wires".\cite{10}  In the early days of wireless communication, such systems were very fragile but today they have become exponentially more resilient. In the simpliest form radio jamming is defined as the transmission of radio signals that disrupt communications by decreasing the signal to noise ratio (SNR) between the transmitter and receiver(s)(need citation).  This jamming can be either deliberate or unintentional.  A common example of unintentional jamming is a microwave oven ironically.  Microwave oven operate with a wavelength of 122 millimetres which translates to 2.45GHz from the equation shown below.  This directly interferes with channels defined under the IEEE 802.11 standard, also known as Wi-Fi(insert citation).  Deliberate jamming on the otherhand, is generally more saphisticated and takes many different forms.\\

\begin{equation}
\lambda=v/f
\end{equation}

Intentional communications jamming is usually aimed at radio signals in a militaristic setting, where consequences are insigificant or out of the relm of the law. In the most rudimentary designs, a jammer will simply tune their own frequency to that of their enemy and with a similar moduation scheme and significant power disrupt the enemies transmissions.  The most common types of this form of signal jamming are random noise, random pulse, stepped tones, warbler, random keyed modulated CW, tone, rotary, pulse, spark, recorded sounds, gulls, and sweep-through(insert citation).  These method obviously or subtly disrupt transmissions by inserting electronmagnetic energy into the transmission space of the receivers.  Mathematically what is occuring is that the jammer is producing randomly chosen data that is non-orthogonal to the data which the friendly transmitter is producing.  Since this jammer's data is pseudo random when his transmissions are added to the enemy's, the result appears to be random as well.  Therefore the signal is unrecoverable.  As mentioned above, the jammer must produce signals that are non-orthogonal to the enemy of his jamming will have no effect.  An example below shows random noise at a significant noise level is added to a previously destinquishable signal.\\

INSERT FIGURE OF Modualated and noisy modulated DBPSK Signals\\

\section{Anti-Jamming}

Anti-jamming has been considerably outlined in the introductory chapter, therefore this section will examine more advanced narrowband and wideband techniques that involve filtering rather than avoidance.  All of these approaches have various monitary costs, constraints, and power limitations.  First of all narrowband mitigation techniques will be considered.  These include adaptive filtering, time-frequency domain filtering, adaptive antennas and subspace processing.  By combining several of the listed techniques wideband jammers can also be address, under certain conditions.  The table below compares these techniques with various attributes.\\

INSERT TABLE\\

Adaptive filtering is a well defined solution in jammer mitigation, but is considerably the most limited.  Most notably the jammer must be relatively narrowband and the period of the jammer must be relatively short.  An example of an adaptive filtering technique is a suppression filter.  Suppression filters assume statistically the signal is gaussian, which results in the optimal filter being linear.  This filter essentially solves the Wiener equation for an optimal filter, but generally a Least Means Squares (LMS) implementation is used instead of inverting the correlation matrix\cite{11}. The matrix inversion of the correlation matrix is considered a zero forcing equalizer and is extremely unstable in the presence of small noise.\\


Time-frequency domain filtering attempts to represent the transform the received signal in such a way that it is possible to easily distinguish the jammer from the data signal.  A Short-Time Fourier Transform (STFT) can be used to accomplish this goal.  A STFT operates by sliding a window across a signal and taking the fast fourier transform (FFT) of that window.  \cite{12} uses the STFT to break a signal into its frequency components, from this information with a narrowband jammer only a small number of frequency domain bins contain nearly all of the interferer.  Therefore these bins can be simply nulled and an inverse FFT is applied to the signal to regain its time domain version.  This is very effective with the use of spread spectrum signal with a narrowband jammer.\\

%\being{equation}
%
%\displaystyle\sum_{n=-\infinity}^{\infinity} n^{2}
%
%\end{equation}

Filter banks is a second methodology that can be used to reduce spectral leakage in the frequency domain, which is a large problem with the STFT approach.  Also filter banks don't inject interference when the jammer isn't present, which is a common problem when the jammer turns on and on.  Filter banks provide jammer suppression after their spectral decomposition stage, since at this point sub-band encoding can be accomplished this spectral modification can become excision for the jammer\cite{13}.  A similar decomposition is the wavelet transform.  The wavelet transform is much more flexible than a STFT because STFT has a fixed resolution for a given FFT size unlike the wavelet transform.  Subspace processing can also be applied in this way.  The jammer subspace can be made to orthogonal to the wanted signal subspace, nullifying the jammer's effects\cite{14}.\\

Besides these signal processing methods, physical techniques can be use to do spatial filtering.  These techiques make uses of several antennas, and as an assumption the number of interferers must be equal to or less than the number of antennas.  The first approach is called Null Steering.  Null steering constantly computes the weights in order to minimize the received energy level. In effect, this technique attempts to steer the antenna away from the jammer.  The second approach is called beamforming.  Beam Forming tries to adjust the antenna in order to maximize the SNR. In effect, the antenna beam is steered in the direction of the desired signal.  It is however, possible to end up in situations where the jammer is in the same direction as the signal source. This is a postcorrelation technique since the desired signal has to be correlated in order to obtain the SNR. Also, prior knowledge of the signal direction and the host location is required(insert citation).\\


All of these approaches historically applied to spread spectrum communication systems because narrowband jammers fundamentally are considerably easier to deal with in this setting.  They are rather straightforward because the jammer effects only a fraction of the transmitter's transmission space.  Therefore when wideband jammers exist many of these schemes fall apart.  Other avenues or scenarios must be considered in such a situation to overcome this limitation.  Before a solution can be considered, additional signal processing and communication theory must be understood.  These topic will be examined in the following sections.\\



\section{Communication Systems}

Modern wireless digital communication systems are based on a rich tradition of analog experimentation and theory.  These technologies surround us on a daily basis from cellphones, car radios, GPS, and many more.  All these of these devices communicate over wireless links and are built upon the same building block of transmission and reception theory.  Many perspective can be taken, but the most generic observation should be taken at the system level.  Depending on the level of saphistication these blocks can expand greatly, but still solve the same issue caused by the wireless transmission of digital access across an envirorment.  Such non-idealities such as frequency offsets, doppler effect, signal echos, phase shifts, and several more.  These must be compensated for to successful receive uncorrupted information.\\ 

Before the receiver, which is the most complicated part of a communication pair, the transmitter must be examine.  The transmitter's primary goal is to send data in a resilient form or structure to create a more managable signal for the receiver.  This is accomplished in several steps, and the function or purpose of the overall system determines the sophistication of the design.  Figure ~\ref{fig:Transmitter_System_Diagram} outlines the major building blocks of the transmitter; consisting of the coder, pulse-shape filter, and frequency translator.\\

INSERT SYSTEM DIAGRAM OF A TRANSMITTER\\

The transmitter's soul purpose is the send data that is convient for the receiver to understand, and allow others to use the transmission medium as well.  The coding phase of the transmitter can have many purposes and features, but simply it will encode data into a symbol with a form of redudancy or scheme that will help the receiver reconstructed the information more easily.  Next the pulse-shape filter is used to help separate data from one another and help maximize the SNR at the receiver.  This filtering can be done with an asortment of filter shapes, but the most popular is the raised square-root cosine filter.  After the pulse-shaping the signal is translated into frequency information and upconverted to a high RF with a carrier signal.  The translation is done with a modulation scheme such a binary phase-shift keying (BPSK) or pulse amplitude modulation (PAM).  The is upconverted by mixing the signal with a sinusoid, seen by equation \eqref{mixing}.  This done because low-frequency signals such as speech, music, or digital data can be much more efficiently transmitted at higher frequencies\cite{9}.  \\

\begin{equation}\label{mixing}

cosine(x)*cosine(y)=(cosine(x+y)+cosine(x-y))/2

\end{equation}


At the system level, a modern digital receiver can be broken down into a small set of distinct categories or operations: carrier sychronization, timing synchronization, equalization, and frame synchronization.  These sections work together in series to provide smooth transmission of data, and many techniques exist within theses categories to accomplish its goal.  In most communication systems, after the radio frequency (RF) front-end, the first operation done on the received signal is frequency compensation and down conversion.  This compensation needs to accomplished because non-idealities and differences exist between the transmitter's and receiver's oscillator.  Therefore this is continually compensated for and corrected.  Carrier recovery can be accomplished using several methods that include but are not limited to: squared difference loops, phase-locked loops, costas loops, and decision-directed phase tracking(INSERT CITATION).\\

After carrier recovery the signal is pulse-shaped with the same filter shape used at the transmitter.  This will help maximize the SNR of the signal.  Then the signal must be correct again for timing.  The problem of timing recovery is to choose the instants at which to sample the incoming signal.  This is generally done through a interpolation mechanism of the transmitted signal.  Since at the transmitter the signal is upsampled to symbols, a single data point or bit is represented by several received data points.  Therefore these points can be interpolated together for a more accurate estimate of the original data.  Timing recovery also can be done with a several methods including: output power maximization, Mueller-Muller method, and decision-directed.\\

After this point the receiver designs can vary greatly, as the design in this thesis will present, because this is where most of the digital signal processing (DSP) will take place.  This section, call Equalization, is responsible to correcting any effect the channel has on the signal. This includes multipath, noise, 	and other distortions that cause intersymbol interference (ISI).  Equalizer implementations are designed to compensated for types of disturbances that occur under certain systems.  The equalizer stage is most often coupled with the frame synchronization stage so the equalizer itself can adapt to changing conditions.  This is known as soft decision making.  Equalizer techniques include but are not limited to: LMS, decision-directed, dispersion-minimizing, viterbi, blind, and turbo equalizers.\\  

\subsection{Equalizers}

Equalizers can be considered the most complicated design of an entire communication system since they combat a series of distortions.  The primary result of these distortions is called intersymbol interference (ISI).  ISI simply means that symbols interact with one another in the channel space and cannot be considered independent from one another.  Since this interference is generally considered a frequency selective disruption or disperation a filter needs to be employed to reverse such effects.  This filter must be adaptable because the channel distortion cannot be know prior to transmission.\\  

As listed in the previous section, many equalizers exists and operate under specific conditions.  Here several linear equalizers will be discussed in detail including maximum-likelihood sequence detection, adaptively trained equalizer, and decision-directed linear equalization.  The goal of all of these equalizers is to find an FIR filter that when convolved with the received signal produced the original transmitted data \( \textbf{\^{x}}=y \asp e\).  The figure below outlines a typical FIR structure for which the equalizer will create the appropriate coefficents \(c_{0},c_{1},..., c_{n} \) for.  These equalizers also examine the condition of an additive white gausian noise (AWGN) channel and uncorrelator or independent interferers.\\

INSERT FILTER STRUCTURE DIAGRAM\\

The Zero Forcing Equalizer (ZFE) uses peak distortion criteria to determine equalizer coefficents.  If \(H_{c}(f)\) is assumed to be the effects of the channel, the ideal equalizer would be \( H_{Eq}(f)=1/H_{c})(f)\).  This can also be consider the inverse of the channel.  The filter coefficents are modeled as weighted pulses convolved with the channel show be the equation below.\\

\[ p_{eq}(t) = \displaystyle\sum_{k=-M}^{M} w_{k}p_{r}(t-kT)  \]

Unfortunately the ZFE has a large disadvantage, it cannot compensate for small amounts of noise.  Technically it will amplify all noise of the received signal, and if any elements of the channel matrix are considerable small then the equalizer becomes unstable. Therefore this is generally considered a more theoretical or elementary equalizer formulation.  To overcome this problem the zero ISI condition must be relaxed allowing for noise which if small can easily be overcome by such operations as quantization or decision making.  The Linear Minimum Mean Squaed Error Filter (LMMSE) takes this relaxation into account.\\ 

The LMMSE assumes that the symbols are uncorrelated with one another and uncorrelated from the noise in the channel.  This approach tries to minimize the mean square error, a common measure of estimator qualities.  The estimator is defined as \( \^{x}_{MMSE}(y)=\e{x|y}  \).  If \(x\) and \(y\) are jointly Gaussian, then the LMMSE will be linear.  This function or equalizer design minimizes the mean square error.  To simplify further an extension to random vectors can be examined.  An estimate can be made for the original vector \(x\) represented by \(\^{x}\), resulting in the linear equation \(\^{x}=Ay+b\).  The LMMSE will minimize the mean square error \(\e{\|x-\^{x}\|^{2}}\).\\

Besides these linear equalizers outlined, an adaptive approach can also be considered.  The LMS or Gradiant algorithm utilizes a traditional technique for minimizing the error in a signal.  This method is historically known as the "Method of Steepest Decent" or "Newton's Method". By calculating the error of each received symbol, this can be fed back into the system for future symbols.  This error with shape the equalizer's filter coefficents to match the inverse of the channel.  The equations are outline below:\\

\[ y[n]=w[n]^{H}F[n]\]
\[ e[n]=A_{n}-y[n]\]
\[ w[n+1]=w[n]+\mu[n]F[n]\] 

In these equations \(\mu\) acts as the algorithm's stepsize determining how quickly it will converge.  It must also be considered that the larger the stepsize the higher the probability it may become unstable.  As long as the channel's effects are slow changing this equalizer can easily maintain up to date estimates while corrupting little of the data as possible.\\  

All of the methods proposed so far require known data to correct against.  This data is called trainging data and generally comes in the form of a preamble in a frame.  The preamble is added to the beginning of each frame so the equalizer can learn from the effects on that specific data.  The preamble is the same for all frames and is always used so the equalizer will always be learning.  But what happens when data is unknown in the frame, for example the data portion of the frame.  This is where blind equalization comes into play.\\

Several blind equalizers exist but an extension of the LMS equalizer for blind situations will be examined here called the decision-direct equalizer.  For a blind equalizer to operate an error generation mechanism must be evaluate, but since the data symbols are unknown, a decision device must be used inplace.  This decision device is a quantization method and error is generated from this quantization.  This error generation is elaborated from the equations below:

\[ e = 1/2\e{(sign(y[k]-y[k])^{2}}\]

This is quite similar to the original LMS implementation except instead of a known symbol the data is quantized using the sign function.  This type of quantization using the sign function is only applicable is binary modulation schemes such as BPSK.  This equalizer method is usually combine with a training equalizer method in practice, since if a nearly closed eye is observed this equalizer cannot open it.\\

In this section we have examined several equalizer technics while outlining their advantages and disadvantages.  Most techniques require some training mechanisms to operate under heavy channel distortion, and blind techiques such as decision-directed equalization will fail under these conditions.  Unforunately such training data can take considerable resources, lower overall data throughput.  In practice as much as 20\% of frame information is training.  Therefore other technique must be considered to help overcome this obsticle.\\


\subsection{Superimposed Training Equalizer}

As mentioned in the previous section, many implementations exist for equalizer designs, but this thesis will examine the effectiveness of superimposed training symbols in frequency selective channels.  In traditional equalizers, channel estimation is achieved through the use of training data or pilot symbols.  These symbols are both known to the transmitter and receiver, providing the basis for an estimate.  In these equalizers all training symbols are placed at the start of a frame,\cite{16} shows that under high SNR training-based schemes are capable of capturing most of the channel capacity, while under low SNR they are highly suboptimal.  Superimposed equalizers try to overcome this problem along with other to provide more optimal estimates.  Superimposed equalizers physically add its training symbols to the data stream instead of concatinating symbols, saving precious bandwidyj bandwidth\cite{16}.  To accomidate such pilots, energy must be shared among the data and hidden pilots\cite{15}.  \cite{19} shows that fora transmitter of fixed power, with an additive pilot sequence the decrease in data signal power is equal to \[ K_{loss}=\frac{E[\|s(k)\|^{2}}{E[\|s(k)\|^{2}]+E[\|u(k)\|^{2}]}\] equivalent to \(10logK_{loss}dB\) in signal to noise ratio (SNR).  Other disadvantages include an increased signal envelope fluctuation that can be undesireable in nonlinear transmit power amplifiers\cite{17}.\\  

At the receiver, channel estimation can be done using several techniques in both the frequency and time domain.  \cite{17} examines a time domain approach for synchronized averaging of the received signal.  It is important to note that this synchronization isn't related to transmitter and receiver synchronization.  \cite{17} and \cite{18} both assume that the signal \( x(n)\) and noise \( v(n) \) have zero mean and \(E[m_{x}(n)] = d(n) = p(n) \ast h(n)\).  Therefore since \(p(n) \) is the known superimposed periodic pilot sequence, \(h(n)\) can be determined.  \(h(n)\) is generally considered frequency selective, and such channels can be quite difficult to deal with esspecially with multipath.  Multipath interference is a distortion cause when copies of the original signal arrive at the receiver delayed ontop of the originally received non-delayed signal.  This delayed signal essential took another path to the receiver, and this interference is commonly called ghosting in such applications as television broadcasts(INSERT CITATION).\\

Superimposed equalizers are able to better compensate for large multipath channels because they can spread their training symbols throughout the signal itself.  This spreading not only provides a spreading in time but also in other dimensions such a frequency.  Therefore if the training symbols are chosen correctly and place correctly, then the can be spread across the frequency spectrum efficently and capture its selectivity.  Before the pilots can be examined, the channel must be defined.  The channel will be of block length \(N\), and the channel is also time invariant across single blocks, but variable across blocks.  The memory of this channel is of maximum length \(L-1\), and the impulse response of the channel is defined as \(\textbf{h}=[h_{0},...,h_{L-1}]^{T}\).  Since there are \(N\) blocks in the channel, the channel matrix \(H\) is modeled as an \(N x N\) circulant matrix, with the received signal as expressed as:

\[ \textbf{x}=\textit{H}\textbf{s}+\textbf{v}  \]

Here \(\textbf{v}\) is assumed to be zero mean white noise.  The vector \(s\) is a combination of known training symbols and unknown data.  The optimal placement for such training is where the channel undergoes nonergodic fading considered here \cite{20}.  \cite{16} continues on to say that optimally, assuming symbols are placed in clusters of length \(\alpha \ge 2L+1\), this scheme is quasi-periodic.  The variable \(\alpha\) represents the cluser size in this scenario.  It is also important to note that this placement makes sure that the training is always orthogonal.\\

Another consideration that must be considered is how these training symbols interfer with the data itself, and is the training symbols dependent on the data or even the modulation scheme.\cite{15} examines this aspect and proposes a solutions that provides a data independence condition.  As explained, since the training data is periodic it can be placed in equispaced frequency bins, while data is spread across all frequency bins.   Therefore the pilot must be design to distort the data vector of the discrete fourier transform is zero.  In the superimposed training data case, this is done by using the cyclic mean of the data.  Therefore all that needs to be done is the removal of the cyclic mean \(\textbf{e}=\textbf{Jw}\).  \(J\) is the kronecker product of an identity matrix and the fractionally spaced locations of the pilot tones.  Therefore at the pilot frequency only the training symbols are visible for the channel estimation.  Formally here is the transmitted result including pilots and data: \[ s= (I-J)w+c  \].\\


In summary, modern research on superimposed training focuses primarily on the training symbol generation for a certain type of communication sytems design from single transmission to MIMO.  Unfortunately little to no physical implementations exists for such system.  This is true because of the sychronization issue that exist when using superimposed training symbols.  Since they are directly placed with transmittion data it can be difficult to determine their locations in a sequence blindly, which is done in real world systems.  This problem must be considered when phyiscal implementations are proposed.\\

\section{Software Defined Radio}

For the past two decades there has been a peradyme shift is the definition of a radio device.  The conversation has to do with the question of where hardware ends and where software begins.  The term Software Defined Radio, coined by Dr. J. Mitola III,  defined as a set of digital signal processing (DSP) primitives, a metalevel system for combining the primitives into communication system functions (transmitter, channel model, receiver, etc.), and a set of target processors on which the software radio is hosted for real-time communications\cite{21}.  Dr Mitola understood how software provided the flexibility that hardware never could, and as time made it more maliable SDR would become dominant.   


%Just a small section for part of the background of my research. \cite{bibtexcite}


%\chapter{Conclusions]
%\label{ch:conclusions}

%This is the end of my paper.

%%%%%%%%%%%%%%%%%%%%%%%%%%%%%%%% THE FINISH %%%%%%%%%%%%%%%%%%%%%%%%%%%%%%%

%\appendix

% BIBLIOGRAPHY STUFF...
%\nocite{*}
%bibliography stuff

%SEE HERE for BIBTEX styles: http://www.cs.stir.ac.uk/~kjt/software/latex/showbst.html

\bibliographystyle{amsplain}
%\bibliographystyle{plain}

\bibliography{mybib}

\end{document}


%useful links:
%latex VS pdflatex : http://www.andy-roberts.net/misc/latex/pdftutorial.html
%basic intro to latex: http://www.andy-roberts.net/misc/latex/latextutorial1.html
%collection of latex links: http://www.andy-roberts.net/misc/index.html
%collection of math symbols: http://www.comp.leeds.ac.uk/andyr/misc/latex/tutorial9/symbols.pdf
%latex matrix examples: http://www.physicsforums.com/showthread.php?t=146358

%lgrind binary for Macintosh:
%https://netfiles.uiuc.edu/galanaki/www/Hints.html
% (place lgrindef in same directory as *.tex, place lgrind in /usr/local/bin

