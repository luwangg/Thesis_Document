%%%%%%%%%%%%%%%%%%%%%%%%%%% START WITH LATEX PREAMBLE %%%%%%%%%%%%%%%%%%%%%%
% reminder: its = possessive
%           it's = it + is

%use [11pt, draft] for draft markings.
\documentclass[11pt]{mvlthesis}
\usepackage[dvipdfm]{graphicx}
\usepackage{verbatim} %,amssymb,amsmath}
\usepackage{epstopdf}
%\usepackage{afterpage}


%%%%%%%%%%%%%%%%%%%%% SET UP ALL THE TITLE PAGE VARIABLES %%%%%%%%%%%%%%%%%%

\title{\scshape \mbox{THESIS OR DISSERTATION TITLE}\\
\scshape \mbox{TITLE LINE 2}}

\author{Travis F. Collins}
\thesis_or_diss{Thesis}
\degree_type{Master of Science}
\field{Electrical and Computer Engineering}
\degreeyear{December 2012}
\chair{Professor Alexander Wyglinski}
\chairtitle{Major Advisor}
\membertwo{Professor Y}
\memberthree{Professor Z}

%%%%%%%%%%%%%%%%%%%%%% INCLUDE USER DEFINED COMMANDS %%%%%%%%%%%%%%%%%%%%%%%

\newcommand{\bi}{\begin{itemize}}
\newcommand{\ei}{\end{itemize}}
\newcommand{\ii}{\item}
\newcommand{\be}{\begin{enumerate}}
\newcommand{\ee}{\end{enumerate}}
\newcommand{\ie}{\item}

\newcommand{\fig}[5]{
    \begin{figure}[#1]
    \begin{center}
    \includegraphics[#2]{#3}
    \end{center}
    \caption{#4}
    \label{#5}
    \end{figure}
}

%\frenchspacing

%%%%%%%%%%%%%% SPECIFY WHICH PARTS OF THE THESIS YOU WANT PRINTED %%%%%%%%%%

\renewcommand{\baselinestretch}{1.5}

\newcommand{\pderiv}[2]{\mbox{$\frac{\displaystyle \partial #1}{\displaystyle \partial #2}$}}

%%%%%%%%%%%%%%%%%%%% Done with setup, document starts here %%%%%%%%%%%%%%%%%
\begin{document}


%%%%%%%%%%%%%%%%%%%%%%%%%%%%%% TITLE + ABSTRACT %%%%%%%%%%%%%%%%%%%%%%%%%%%%
\maketitle
\begin{abstract}

Since the advent of modern digital communications in the 20th century there has been an explosion in the demand for wireless spectrum. As a result spectrum is becoming an increasingly scare resource. This demand is a direct result of the availability and relatively inexpensive cost of such wireless device. Therefore in such environments as militaristic theatres the probability of interfering transmissions, intended and unintended, has steadily grown to a point where techniques need to be consider to combat such occurrences. More directly, in such situations when interfering signals are partially or completely understood measures can be taken to overcome such difficulties. Under these assumptions several well know techniques are can be applied to combat such scenarios.  This research analyzes the feasibility of combining Antenna Subset Selection, Spectral Subtraction, and Blind Source Separation signal processing techniques to accomplish this goal.  Together they provide multiple avenues of signal separation to remove such jamming effects, for both narrow and wide bands, without hindering the mobility of the nodes themselves.


\end{abstract}

%%%%%%%%%%%%%%%%%%%%%% ACKNOWLEDGMENTS + TABLE OF CONTENTS %%%%%%%%%%%%%%%%%

\begin{frontmatter}

\begin{acknowledgements}
%\begin{center}
%\vspace{0.4in}
%\end{center}
\end{acknowledgements}
\tableofcontents
\listoffigures
\listoftables

\end{frontmatter}

%%%%%%%%%%%%%%%%%%%% INCLUDE THE REST OF THE DOCUMENT %%%%%%%%%%%%%%%%%%%%%%


\chapter{Introduction}
\label{ch:introduction}
\section{Motivation}

Since the advent of modern digital communications in the 20th century there has been an explosion in the demand for wireless spectrum.  As a result spectrum is becoming an increasingly scare resource(Insert citation).  This demand is a direct result of the availability and relatively inexpensive cost of such wireless device.  Therefore in such envirorments as militaristic theatres the probability of interfering transmissions has steadily grown to a point where techniques need to be consider to combat such occurrences.  More directly, in such situations when interfering signals are partially or completely understood measures can be taken to overcome such difficulties.\\

In military theatres it is extremely common to observe friendly operated high-power broadband jamming signals(citation).  Such devices exists as part of group convoys in several branches of the military and in many other forms in contested territories or war-zones.  Unfortunately such devices block both friendly and hostile communications, and current anti-jamming techniques haven't provided a viable solution to this problem.  Therefore new avenues should be considered, utilizing more flexible radio technologies.\\

Understanding how to overcome such challenges is a complex task; with vastly different transmission envirorments and differing operating devices and operating standards.  A new system that could combat such downfalls should rely on all friendly information, or be able to construct solutions of its own from a set of tools given to the radio.  Such tools should be flexible and easily modified, changed, or improved.  This ability to easily change or adapt is a key feature as the technical requirements can change from day to day, or between branches of the military itself. As such a solution should have the following attributes:

\begin{itemize}
\item \textbf{Flexible:}

\item \textbf{Resilient:}

\item \textbf{Hardened:} in changing enviroments

\end{itemize}


\section{State of the Art}

Current implementations in anti-jamming technology lies on the strateling point of hardware and software in the communications world.  This is true because hardware provides the speed and performance needed for digital data transmission, while software provides higher level intelligence and flexibility in such layers as the media access control layer and the network layer of the OSI model (insert citation of OSI model.)\\

Insert figure of OSI model\\




Types of anti-jamming attacks

Narrow focus of problem to single narrow band jammer, with no mobility.

\section{Thesis Contributions}

\section{Thesis Organization}

\chapter{Background}
\label{ch:background}

Here is some background you'll need to know about my research.

\section{Background subsection}

Just a small section for part of the background of my research. \cite{bibtexcite}


%\chapter{Conclusions]
%\label{ch:conclusions}

%This is the end of my paper.

%%%%%%%%%%%%%%%%%%%%%%%%%%%%%%%% THE FINISH %%%%%%%%%%%%%%%%%%%%%%%%%%%%%%%

%\appendix

% BIBLIOGRAPHY STUFF...
%\nocite{*}
%bibliography stuff

%SEE HERE for BIBTEX styles: http://www.cs.stir.ac.uk/~kjt/software/latex/showbst.html

\bibliographystyle{amsplain}
%\bibliographystyle{plain}

\bibliography{mybib}

\end{document}


%useful links:
%latex VS pdflatex : http://www.andy-roberts.net/misc/latex/pdftutorial.html
%basic intro to latex: http://www.andy-roberts.net/misc/latex/latextutorial1.html
%collection of latex links: http://www.andy-roberts.net/misc/index.html
%collection of math symbols: http://www.comp.leeds.ac.uk/andyr/misc/latex/tutorial9/symbols.pdf
%latex matrix examples: http://www.physicsforums.com/showthread.php?t=146358

%lgrind binary for Macintosh:
%https://netfiles.uiuc.edu/galanaki/www/Hints.html
% (place lgrindef in same directory as *.tex, place lgrind in /usr/local/bin

